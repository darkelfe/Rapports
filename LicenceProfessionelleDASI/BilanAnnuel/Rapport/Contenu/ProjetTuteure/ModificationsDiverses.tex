\chapter{Modifications diverses}
\section{Intégration des NVOCC}
Après la phase des tests et la mise en place de l'envoi FTP, je suis resté plusieurs semaines à travailler sur d'autres projets de \solulog. Mais au bout d'un moment, nous avons reçu une demande d'un de nos clients, demandant s'il serait possible d'adapter notre module d'envoi de BL pour qu'il puisse traiter les \og NVOCC\footnote{NVOCC : Non-Vessel-Operating Common Carrier, entreprises s'avoisinant aux compagnies maritimes.} \fg.

~

Après avoir consulté la documentation d'INTTRA avec Samuel, nous avons appris que le cas des NVOCC était déjà géré par la norme IFTMIN. Il a donc été décidé que j'adapterais mon module (et pas celui en PHP) pour le rendre compatible avec les NVOCC.

~

Fondamentalement, les NVOCC sont très proches des compagnies maritimes. L'EDI des BL fait néanmoins une distinction au niveau de plusieurs segments et variables. Certaines, comme le numéro de container, ne sont plus obligatoires alors que d'autres segments le deviennent.

~

Ma première action a été de nettoyer mon code. Je l'ai fait pour plusieurs raisons. La première est liée à la demande des NVOCC. Les modifications à apporter au code sont minimes et ne portent que sur les tests obligatoires ou non. J'ai donc adapté mon code pour qu'il puisse fonctionner sans distinction avec une compagnie maritime ou avec une NVOCC.

~

Le seul endroit où j'ai dû opérer des modifications en profondeur fut la fonction gérant les tests. Avec l'aide de la documentation, qui listait les différences entre les deux, j'ai créé une fonction qui contenait l'ensemble des tests obligatoires communs aux deux. Ensuite, j'ai créé une fonction pour les tests obligatoires des compagnies maritimes, et une autre pour ceux des NVOCC. Puis, selon le type de \og compagnie \fg, j'exécute la fonction correspondante.

La suite du code étant également identique pour les deux, je l'exécute lui aussi sans tenir compte du type.


\section{Modes de fonctionnement}
J'ai profité de cette mise à jour du code pour y intégrer quelques nouvelles notions liées au mode d'exécution du module. En fait, celles-ci étaient déjà présentes dans \pireus, mais je ne les avais pas réécrites car elles avaient été jugées inutiles sur le moment. Mais finalement, nous avons décidé de les mettre en place.

~

Comme je le disais, j'ai implémenté deux moyens d'influencer le comportement du module lors de son exécution. Le premier consiste en un choix entre un mode dit \og de production \fg{} et un mode \og debug \fg. Le mode de production constitue le fonctionnement habituel du module : les tests sont faits, le message généré puis envoyé par FTP. C'est le mode généralement actif chez les clients, à de très rares exceptions près. Le mode debug, quant à lui, effectue le travail jusqu'à la génération du message, mais ne l'envoie pas à INTTRA. Il s'agit typiquement d'un mode adapté aux tests et au développement.

~

Le second moyen est un choix utilisateur entre les tests seuls et l'envoi. Dans le premier cas, le module effectue seulement les tests sur les valeurs saisies, mais n’envoie pas le message. Dans le second cas, le message est transmis à INTTRA. Au départ, il ne s'agissait pas d'une fonctionnalité voulue, mais l'utilisation de \pireus{} a démontré que les utilisateurs souhaitaient avoir deux boutons séparés, l'un uniquement pour les tests et l'autre pour l'envoi.

~

L'implémentation de ces modes de fonctionnement est très simple. Elle se fait au sein de la fonction principale, qui appelle les étapes de la génération d'un BL les unes après les autres : lecture, tests, écriture, etc.

Selon le mode de fonctionnement, la fonction s'interrompt après sa dernière action. Par exemple, le mode debug s'arrête après la création du message.