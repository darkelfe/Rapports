\documentclass[10pt]{report}

\usepackage[utf8]{inputenc}
\usepackage[T1]{fontenc}
\usepackage[francais]{babel}

\usepackage{lmodern}

\usepackage[top=2cm,bottom=1.25cm,right=.75cm,left=2cm]{geometry}

\usepackage{color}
\usepackage{listings}
\lstset{ %
	language=HTML,					% choose the language of the code
	basicstyle=\footnotesize,		% the size of the fonts that are used for the code
	numbers=left,					% where to put the line-numbers
	numberstyle=\footnotesize,		% the size of the fonts that are used for the line-numbers
	stepnumber=1,					% the step between two line-numbers. If it's 1 each line will be numbered
	numbersep=5pt,					% how far the line-numbers are from the code
	backgroundcolor=\color{white},	% choose the background color. You must add \usepackage{color}
	showspaces=false,				% show spaces adding particular underscores
	showstringspaces=false,			% underline spaces within strings
	showtabs=false,					% show tabs within strings adding particular underscores
	frame=single,					% adds a frame around the code
	tabsize=4,						% sets default tabsize to 2 spaces
	captionpos=b,					% sets the caption-position to bottom
	breaklines=true,				% sets automatic line breaking
	breakatwhitespace=false,		% sets if automatic breaks should only happen at whitespace
	title=\lstname,					% show the filename of files included with \lstinputlisting
	escapeinside={\%*}{*)},			% if you want to add a comment within your code
	morekeywords={*,...}			% if you want to add more keywords to the set
}

\usepackage{hyperref}

\title{Projet tuteuré en licence professionnelle\\Compte-rendu de synthèse de la formation}
\author{Julien \bsc{Rosset}}
\date{\today}
\hypersetup {
	pdftitle	=	{Synthèse et bilan final de la formation - Licence Professionnelle DASI en alternance},
	pdfauthor	=	{Julien ROSSET},
	pdfsubject	=	{Compte-rendu de synthèse en licence professionnelle DASI en alternance},
	pdfkeywords	=	{Compte-rendu, synthèse, licence professionnelle},
	pdflang		=	fr_FR,
	pdfborder	=	0 0 0,
	frenchlinks	=	true
}

\usepackage{fancyhdr}
\pagestyle{fancy}

\fancypagestyle{plain}{%
\fancyhf{}
\fancyhead[L]{Julien \bsc{Rosset}}
\fancyhead[C]{Synthèse et bilan final de formation}
\fancyhead[R]{\today}
\fancyfoot[L]{Université Claude Bernard Lyon 1 - IUT A - Département Informatique - LP DASI A}
\fancyfoot[R]{\thepage}
\setlength{\headsep}{10pt}
\renewcommand{\headrulewidth}{1px}
\renewcommand{\footrulewidth}{1px}}

\fancyhf{}
\fancyhead[L]{Julien \bsc{Rosset}}
\fancyhead[C]{Synthèse et bilan final de formation}
\fancyhead[R]{\today}
\fancyfoot[L]{Université Claude Bernard Lyon 1 - IUT A - Département Informatique - LP DASI A}
\fancyfoot[R]{\thepage}
\setlength{\headsep}{10pt}
\renewcommand{\headrulewidth}{1px}
\renewcommand{\footrulewidth}{1px}

% Pour la page de garde
\usepackage{color}
\usepackage{changepage}
\usepackage{graphicx}

% Commandes personalisés
\newcommand{\solulog}{\bsc{Solulog}}
\newcommand{\fidit}{\bsc{Fidit}}
\newcommand{\integrale}{\og Intégrale \fg}
\newcommand{\pireus}{\bsc{Pireus}}
\newcommand{\vb}{Visual Basic}
\newcommand{\pdfcreator}{\bsc{PDFCreator}}

\begin{document}
	%\maketitle
	\begin{titlepage}
	\changepage{+2.5cm}{+5cm}{+0cm}{-2.5cm}{+0cm}{-1cm}{+0cm}{+0cm}{+0cm}

	\begin{center}
		\colorbox{black}{\parbox[top][1.5cm][c]{\textwidth}{
			\begin{center}
				\textcolor{white}{\sffamily{\bfseries{\LARGE{Licence Professionnelle DASI en alternance}}}}
			\end{center}
		}}\\[0.75cm]

		\textsc{Septembre 2010 - Septembre 2011}\\[0.75cm]
		Tuteur enseignant: M. Jean-Michel \bsc{Martiniere}\\[1cm]

		\textsc{\huge \bfseries Synthèse de formation}\\[1cm]

		\emph{Apprenti: }\\Julien \textsc{Rosset}\\[1cm]

		\fbox{\parbox[top][2cm][c]{\textwidth}{\begin{center}\Huge \bfseries Synthèse et bilan final de la formation \end{center}}}\\[1cm]


		\begin{figure}[!htb]
			\begin{center}
				\includegraphics[scale=.75]{Contenu/Images/Logo_Solulog.png}
			\end{center}
		\end{figure}
		\fbox{\parbox[top][3cm][c]{\textwidth}{\begin{center}\begin{Large}\bsc{Solulog}\end{Large}\\~\\37, rue Montgolfier\\38090 Villefontaine\\
		\vspace{0.5cm}Maître d'apprentissage: M. Philippe \bsc{Perisse}\end{center}}}

		\vfill

		\hrule
		\begin{center}
			Université Claude Bernard - Lyon 1, IUT A
			\begin{figure}[!htb]
				\begin{center}
					\includegraphics[scale=.25]{Contenu/Images/Logo_IUTA.png}
				\end{center}
			\end{figure}
			Département Informatique\\
			43, Boulevard du 11 novembre 1918 - 69622 Villeurbanne CEDEX\\
			Tél: 04.72.69.21.90
		\end{center}
	\end{center}
\end{titlepage}

	\chapter*{Remerciements}
\addcontentsline{toc}{chapter}{Remerciements}
En premier lieu, je souhaite remercier toute l'équipe des entreprises \solulog{} et \fidit{} pour le merveilleux accueil qu'elles m'ont offert et leurs efforts incessants pour m'intégrer.

Je voudrais également leur exprimer ma gratitude pour m'avoir prêté attention et mis en \oe{uvre} certaines de mes idées afin d'améliorer la qualité du code, et du logiciel d'une manière plus générale.

~

Je souhaite tout autant remercier mes tuteurs, tant à l'IUT qu'en entreprise, pour leur suivi et leurs conseils judicieux.

~

Du côté technique, je tiens à remercier Samuel \bsc{Raposo}, qui a partagé son expérience du module PHP traité dans mon projet tuteuré, ainsi que pour ses nombreuses explications sur l'utilisation des techniques de programmation propres à \solulog.

Je veux également remercier toute l'équipe de développement de \fidit, pour m'avoir si bien accueilli dans leur milieu.


	\renewcommand{\contentsname}{Sommaire}
% 	\setcounter{tocdepth}{3}
	\tableofcontents

	\chapter{Introduction}
bla bla bla

	\chapter{Présentation}
\section{Cadre}
\subsection{Le document}
Comme je l'ai expliqué plus haut dans ce document, les entreprises qui organisent le transport international doivent envoyer de nombreux documents à différentes entités. Nous pouvons citer, par exemple, le document envoyé à la douane et qui détaille les marchandises transportées, celui qui permet de réserver une place à bord d'un transport (avion, bateau, ...) pour les marchandises souhaitées, etc.

~

Dans le cadre du transport international \textbf{maritime}, il existe un document nommé \emph{Boat Landing} (généralement abrégé en \og BL \fg) qui doit être transmis à la compagnie maritime en charge du transport. Ce document renseigne sur les marchandises transportées ainsi que leur répartition dans les différents conteneurs\footnote{Conteneur : il s'agit ici de conteneurs maritimes, aussi appelés \emph{containers}, c'est-à-dire des caissons métalliques destinés au transport de marchandises sur les bateaux. Il en existe de différentes tailles, généralement exprimées en pieds.}. Il permet également de transmettre des informations relatives aux poids et volumes respectifs des conteneurs. Des données se rapportant au client final et aux différents acteurs du transport (banque, destinataire, etc.) y sont également inclus.


\subsection{INTTRA}
Afin de limiter la prolifération de formats différents pour ce document (ainsi que d'autres, non traités dans ce rapport), un grand nombre de compagnies maritimes du monde entier se sont regroupées pour créer une association basée au Danemark : \emph{INTTRA}. Cette dernière a pour principal objectif la standardisation du format des documents qui doivent être transmis aux compagnies maritimes.

~

De nos jours, son rôle va bien plus loin : tout document devant être transmis à une compagnie maritime a pour \textbf{obligation} de passer par INTTRA, qui vérifie alors la validité du document. S'il est correct, il est retransmis à la compagnie maritime ; sinon, un e-mail est envoyé au client à l'origine du message pour lui indiquer les informations manquantes ou les divers problèmes.

~

\'Etant donné qu'un très grand nombre de documents doit transiter par les serveurs d'INTTRA, le traitement des fichiers n'est pas instantané. C'est pourquoi un système de \og statuts \fg {} a été développé. Il s'agit d'une feuille, dont le contenu est lui aussi normalisé, mise à disposition du client et indiquant le statut du document :
\begin{itemize}
	\item \textbf{En attente} : le document n'a pas encore été traité ;
	\item \textbf{Accepté} : le document est valide et a été transmis à la compagnie maritime ;
	\item \textbf{Refusé} : le document reçu contenait des erreurs, il faut se référer au mail correspondant pour obtenir les détails.
\end{itemize}


\subsection{Contenu}
Un fichier-exemple de BL est disponible dans l'annexe \ref{ex_bl}, page \pageref{ex_bl}.

~

Les BL sont en théorie constitués d'une unique ligne, mais l'exemple a été découpé par segments pour simplifier sa visualisation. Un segment est un fragment de fichier de BL véhiculant des informations précises. Au sein du document, ils sont séparés par une apostrophe (visible à chaque fin de ligne dans l'exemple).
\vfill


\section{Mise en \oe{uvre} par \solulog}
Les BL sont rédigés à l'aide d'un EDI\footnote{EDI : \emph{Electronic Data Interchange}, ou \emph{\'Echange de données informatisé}, désigne une \og norme \fg{}sur le contenu et le classement des données dans d'un fichier.} nommé \og IFTMIN \fg. Tout comme un grand nombre d'EDI, il est absolument inintelligible pour un être humain. Puisque seul un ordinateur était capable de l'écrire en un temps raisonnable, \solulog{} a créé un nouveau  module qui serait chargé de le faire puis d'envoyer le document à INTTRA.

~

Alexandre a été chargé du développement du module. \'Etant plus à l'aise avec le PHP qu'avec le Visual Basic, et le module final devant être plus tard vendu à d'autres entreprises, il a décidé de créer un module indépendant qui fonctionnerait comme un web service (utilisation de SOAP).


\subsection{Processus interne}
Tout web service a besoin d'un client pour être exploité. Dans le cas présent, il s'agit de \integrale, le travail commence donc par là. Les données recueillies auprès de l'utilisateur subissent quelques tests. L'absence de certains champs vides est notamment testée, puis les données sont placées dans des variables créées par le web service via une DLL\footnote{DLL : \emph{Dynamic Link Library}, fichier Microsoft Windows contenant des types de données et/ou des fonctionnalités communes à plusieurs applications.}. Ces variables sont alors transmises au web service : \pireus.

~

Après réception des données, \pireus {} commence à faire l'ensemble des tests nécessaires à l'écriture d'un fichier correct. Les données pouvant, en théorie, venir d'ailleurs que de \integrale, les tests sur celles-ci sont également réalisés à nouveau. Une fois les tests terminés et validés, les données sont transformées (les raisons de cette étape seront détaillées en page \pageref{transformation_donnees}, à la section \ref{transformation_donnees}). La rédaction du document peut alors commencer.

Pour finir, le document est transmis à INTTRA via un service FTP\footnote{FTP : \emph{File Transfert Protcol}, un protocole Internet pour l'échange de fichiers.}.
	
	
\section{Objectifs du projet}
\subsection{Problématique}
A mon arrivé à \solulog, le module \pireus~est complet et parfaitement fonctionnel depuis presque cinq ans. Mais il se pose un problème de taille : tant qu'Alexandre était présent au sein de l'entreprise, il pouvait maintenir et faire évoluer le module. Mais depuis sa démission, \solulog{} a perdu la seule personne qui soit capable d'en comprendre le langage. Malheureusement, bien qu'il soit très bien commenté, cela ne suffit pas à contre-balancer la complexité du PHP pour un novice. Par conséquent, l'évolution du module est bloquée.

~

Même si un module fonctionne parfaitement depuis longtemps, il peut arriver qu'il se produise un problème, insoluble sans de solides connaissances du langage. Une fois le problème survenu, il existe peu de solutions. Soit \solulog{} recrute un expert en PHP pour s'occuper du module, soit il faut réécrire le module dans un autre langage.

La première solution pose plusieurs problèmes : un expert en PHP n'est pas forcément formé à des développements dans d'autres langages, ce qui signifierait l'engager pour s'occuper seulement de \pireus. À moins d'avoir des changements radicaux à apporter au module, ce serait sans doute une perte de temps. De plus, la problématique liée au départ d'Alexandre risque fortement de se poser à nouveau lorsque l'expert partira à son tour.

Cependant, la seconde solution n'est pas parfaite non plus : réécrire un module complet, surtout aussi poussé que l'est \pireus{}, implique au préalable une période d'étude du projet. Il s'agit de temps et d'argent qui pourraient être employée dans d'autres projets.

~

Mon supérieur a finalement opté pour la deuxième méthode. Cela a par ailleurs permis de régler un autre problème, inhérent à l'utilisation d'un web service : il est difficile de déterminer si l'erreur provient du client ou du web service lui-même. De plus, malgré ce qui avait été prévu au départ, le module n'a jamais été commercialisé. Cela justifie un peu plus l'emploi de cette solution.

~

Il ne restait plus à résoudre que la question du PHP : d'accord pour réécrire le module, mais dans quel langage ? M. \bsc{Perisse} a choisi le plus simple : le \vb.
\vfill


\subsection{Le projet tuteuré}
C'est à ce moment-là que j'interviens. Le \vb{} étant le premier langage que j'ai appris, je l'ai beaucoup pratiqué. Même si ces connaissances avaient commencé à se dissiper ces derniers temps, je m'en souviens encore et cela revient vite. En outre, j'ai des connaissances basiques en PHP pour m'aider à plonger au c\oe{ur} de \pireus.

Mes connaissances en \vb{} me seront également utiles pour résoudre le problème de \og main d'\oe{vre} \fg. Mais engager du personnel a un coup non négligeable pour l'entreprise. Au final, ce sont trois alternants qui sont recrutés : moi-même pour le développement, et deux commerciales pour aider les finances de \solulog.

~

Pour finir, voici en résumé mon projet tuteuré : transformer un module PHP écrivant des fichiers à l'aide un EDI, en une extension codée en \vb{} du progiciel \integrale{}.

	\part{Synthèse : de septembre 2010 à avril 2011}

\chapter{Introduction}
bla bla bla
% \chapter{Le projet tuteuré}
Bien qu'il ai déjà fait l'objet d'un compte rendu détaillé de réalisation, je ne peux pas faire une synthèse de mon année sans le mentionner.

~

Bien évidemment, ce chapitre à pour but de donner seulement un aperçu de mon projet tuteuré. Si vous souhaiter un savoir plus à son propos, je vous conseille de vous référer au compte rendu détaillé qui lui est consacré. Attention, ledit rapport est dit technique, c'est-à-dire qu'il s'adresse à un public familier de l'informatique et du développement d'application. Il est disponible ici : \url{https://github.com/downloads/darkelfe/Rapports/LicenceProfesionnelleDASI_ProjetTuteure_Rapport.pdf}.

\section{Présentation}
\subsection{Origine du module}
Parmi les nombreuses fonctionnalités proposées par \integrale, il existe plusieurs possibilités d'envoi automatique de documents. En effet, l'organisation du transport international nécessite l'envoi de divers documents officiels, par exemple à la douane, ou à la compagnie maritime pour réserver des emplacement sur un bateau, etc. A l'ère de l'informatique, tous ces documents sont désormais électroniques mais un problème majeur reste : il peuvent être très compliqués à écrire.

C'est pour cette raison qu'\integrale{} propose d'écrire automatiquement ces documents, en y incluant toutes les informations nécessaires, et de les envoyer à son destinataire par un simple clic sur un bouton.

~

Dans le cadre du projet tuteuré, nous devons nous concentrer sur un seul de ces documents : le \emph{Boat Landing}, généralement abrégé en
\og BL \fg. Il s'agit d'un fichier émit à l'attention de la compagnie maritime qui détaille le contenu d'un projet : nombre et propriétés des conteneurs\footnote{Ici, il est fait référence aux conteneurs \emph{maritimes}, c'est-à-dire de \og grandes boites rectangulaires métalliques \fg.}, poids et volume des marchandises qui sont dedans, etc.

Une des particularité de ce fichier, c'est sa totale illisibilité par un être humain. Les informations qu'il contient doivent respecter une norme particulière qui ne facilite pas la lecture.

~

Pour simplifier le travail de nos clients, \solulog{} à donc décider de créer un module qui crée automatiquement ce fichier puis l'envoi par internet.

\subsection{\pireus}
Tout commence donc en 2006 avec Alexandre, qui est chargé de ce travail. Vu que \solulog{} souhaite pouvoir vendre ce futur module indépendamment de \integrale, il est décidé que le module, nommé \emph{\pireus}, sera écrit en PHP, avec seulement un petite partie en \vb{} pour transmettre les données nécessaires à la création du fichier.

~

Quelques temps plus tard, le module est prêt et mis en service. Les années passent et en 2009, Alexandre, le responsable du module et seul développeur, quitte \solulog. \pireus fonctionnant parfaitement, les choses restent comme cela.

Mais à la fin de la même année, \solulog{} se trouve dans l'obligation de modifier ce module pour y intégrer de nouvelles fonctionnalités. Or, il y a un problème assez gênant : personne, au sein de \solulog, ne maitrise le PHP. Finalement, c'est Samuel qui se charge des modifications. Malgré l'aide d'Alexandre, il a beaucoup de difficultés pour appréhender la totalité du module et finalement effectue seulement les modifications demandées.

\vfill

\subsection{Objectifs}
C'est cet événement qui va pousser \solulog{} à réfléchir au problème. Il va finalement en ressortir que le module doit être entièrement réécrit en \vb{} et intégré à \integrale, les tentatives de vente ayant été abandonnées.

~

Mon arrivée en 2010 leur donne l'occasion d'effectuer cette tâche. Je suis donc chargé, à titre de projet tuteuré, de réécrire entièrement le module PHP en \vb. Je dois également en profiter pour écrire une documentation complète sur mon module, afin de simplifier sa compréhension et les possibles modifications ultérieures.

\section{Déroulement du projet}
Mon projet tuteuré s'est déroulé en deux grandes parties, que je vais détailler ici.

\subsection{Phase 1 : écriture du nouveau module}
La première partie, qui se passe sur l'ensemble du mois d'octobre et de novembre 2010, à eu pour objectif principal de créer le nouveau module.

~

Cette première phase est elle-même découpé en sept tâches successives.

\subsubsection{Analyse de l'existant}
Au tout début, avant même de penser à réécrire \pireus, il faut commencer par savoir de quoi traite ce module et comment il est construit. J'ai donc commencé par l'étudier et observer la manière dont est organisé son contenu (fichiers sources, etc.). Ensuite, je me suis lancé dans l'étude du code PHP lui-même. J'ai eu beaucoup de mal, notamment en l'absence d'une documentation claire et l'utilisation de fonctions que je ne connaissait pas encore. Mais j'ai au final réussi à identifier la majorité des données fournies par l'utilisateur dans le fichier final.

\subsubsection{Récupération des valeurs}
Les valeurs fournies par l'utilisateur sont stockées dans la base de données. La première chose que j'ai donc dû faire dans mon nouveau module, nommé \emph{BL\_INTTRA}, ça a été de lire ses valeurs et de les placer en mémoire, afin qu'elles soient facilement accessible par la suite. J'ai eu quelques désaccords avec mon supérieur sur la manière de stocker ces valeurs en questions, mais nous sommes finalement parvenu à un compromis.

\subsubsection{Tests des valeurs}
Les valeurs fournies par l'utilisateur peuvent contenir des erreurs, des éléments incompatible avec les BL ou il peut également manquer des informations essentielles. Donc, avant de générer le fichier de BL final, j'ai dû tester toutes les valeurs lues. Heureusement pour moi, ces tests était déjà présent dans le module d'Alexandre. J'ai donc juste eu à les convertir en \vb{} et les intégrer à mon module. J'ai tout de même réorganisé l'ordre des tests pour que ceux-ci soient groupés par le type des données sur lesquels ils portaient.

\subsubsection{Génération du BL}
A ce moment là, tout est prêt pour générer le message de BL. Avec l'aide de \pireus{} et de la documentation officielle de la norme \emph{IFTMIN D99B}\footnote{Il s'agit de norme qui régit le contenu des fichiers de BL.}, j'ai entrepris la longue et difficile tâche de générer le BL. Cette partie n'a pas été sans mal. J'ai eu de nombreux problèmes, notamment pour calculer certaines variables essentielles au fichier final (poids des marchandises, numéro de révision, etc.), mais j'ai finalement réussi à m'en sortir.

\subsubsection{Écriture d'une documentation}
Comme il était prévus initialement, j'ai dû écrire une documentation portant sur la génération d'un BL. J'ai profité du fait que les segments\footnote{Un segment est un fragment d'un fichier de BL. Il sont séparé par une apostrophe : \og ' \fg.} et leurs particularités étaient encore frais dans mémoire pour réaliser celle-ci. Une première partie de la documentation détaille l'ensemble des tests à mener sur les valeurs avant d'écrire le BL. Je détaille donc les conditions des tests, plusieurs n'étant obligatoire que dans certains cas précis. La valeur à tester ainsi que les possibles tests qui peuvent en découler peuvent dépendre d'une valeur particulière, comme le Brésil pour le port de destination.

La seconde partie de la documentation porte sur les segments : liste et ordre d'apparition dans le BL, mais également lesquels sont obligatoires, les valeurs qu'ils contiennent ou encore si ces valeurs sont elles-mêmes obligatoires.

~

Voici deux exemples de segments issus de ma documentation :
\begin{figure}[h!]
	\begin{center}
		\includegraphics[scale=.54]{Contenu/Synthese_SeptembreAvril/Images/Segment_TDT.png}
	\end{center}

	\caption{Détails du segment \og TDT \fg}
	\label{segment_TDT}
\end{figure}
\begin{figure}[h!]
	\begin{center}
		\includegraphics[scale=.54]{Contenu/Synthese_SeptembreAvril/Images/Segment_NAD_CA.png}
	\end{center}

	\caption{Détails du segment \og NAD-CA \fg}
	\label{segment_NAD_CA}
\end{figure}

\subsubsection{Tests du module}
Une fois la génération des BL fonctionnelle, il a fallu la tester. Il y a trois types de tests différents successifs. Le premier test a consisté à vérifier le fichier obtenu par un outil de test fourni par la norme : \og EDI Test Tool Kit \fg.

Ensuite, avec de fausses données entrée manuellement, j'ai comparé mes résultats avec ceux de \pireus, portant bien sûr sur les mêmes données. Cela m'a permis de corriger plusieurs oublis de ma part.

Et pour finir, mon module à été installé sur le serveur de test d'un de nos client. A partir de ça, j'ai pu faire fonctionner \emph{BL\_INTTRA} sur de véritables données. Et j'ai également pu tester les résultats en les comparant aux fichiers générés chez le client.

\subsubsection{Envoi par FTP}
Il s'agit de la partie la plus simple et la plus courte de mon projet tuteuré : envoyer le BL généré à l'organisme correspondant par internet, par le biais du protocole FTP. Par chance, Alexandre avait déjà réalisé par le passé un ensemble de fonctions qui simplifient à l'extrême l'envoi et la récupération de fichier par FTP. J'ai donc simplement réutilisé ces fonctions pour faire mon envoi.

\subsection{Phase 2 : Intégration et évolution}
La seconde partie de mon projet tuteuré c'est déroulé plus tard dans l'année, à la jonction entre le mois de décembre 2010 et janvier 2011. Ma principale mission était d'intégrer définitivement mon module à \integrale{} et d'y ajouter quelques fonctionnalités.

\subsubsection{Récupération des statuts}
Lorsqu'un client envoi un BL, la société réceptrice met à sa disposition un autre document qui permet de connaître l'état du BL : en cours de traitement, validé ou refusé. \pireus{} se chargeant déjà d'aller récupérer ces statuts par le passé, j'ai également dû le faire. Le traitement est au final très simple : aller récupérer les statuts par FTP, puis de les analyser. J'ai profité de cette réécriture pour simplifier le processus du module d'Alexandre. En effet, celui-ci extrayait beaucoup d'informations inutile du message de statut. Dans mon module, je me contente de récupérer le statut lui-même et d'avertir l'utilisateur de celui-ci.

\subsubsection{Intégration}
Dans un premier temps, j'ai terminé de placer mon module dans \integrale. À ce moment là, mon module était déjà physiquement présent dans le logiciel. De même, appuyer sur le bouton de génération provoquait un appel à mon module. Mais étant donné que \emph{BL\_INTTRA} était tout neuf, \solulog{} n'a voulu l'intégrer qu'a la seule condition qu'il y ai une solution de repli : si mon module échoue à générer un BL, que le client puisse facilement utiliser \pireus. J'ai donc du mettre en place un petit système de choix. Chaque utilisateur, dans son paramétrage peut décider d'utiliser mon module ou celui d'Alexandre. Ensuite, quand un utilisateur clique sur le bouton de génération d'un BL, \integrale{} appel le système correspondant.

\subsubsection{Recette}
Bien que cette étape ai été prévu pour plus tard dans l'année (au alentour du mois d'avril), la mise en place chez le client de module s'est produite bien plus tôt que prévus suite à un appel d'un client. En effet, celui-ci avait tenté de générer un BL contenant trente-quatre conteneurs. Le problème est que la norme impose un maximum de trente marchandises par conteneur et cent conteneurs maximum par BL. Or, au grand regret de notre client, \pireus{} avait un défaut de conception qui avait pour conséquence d'inverser ces deux limites. Donc impossible pour notre client de générer le BL. Face à l'urgence de la situation, \solulog{} a décidé de mettre en place mon module directement chez le client. C'est donc ainsi que la recette de mon module a démarré.

\subsubsection{Intégration des NVOCC}
Le dernier point de mon projet tuteuré était d'ajouter à mon module une fonctionnalité absente de \pireus{} : la gestion des \emph{NVOCC}. Les NVOCC sont des compagnies maritimes un peu particulières, qui n'ont pas exactement les mêmes exigences qu'une compagnie maritime standard. Cela influe sur les tests qui doivent être effectués lors de la génération d'un BL. Donc, via la documentation officielle des BL à propos des NVOCC, j'ai modifié mes tests en fonction du type de compagnie maritime (standard ou NVOCC). J'ai également groupé d'un côté les fonctions propres aux NVOCC et le reste d'un autre côté.

% \subsection{Factures en PDF}

\begin{frame}
	\frametitle{Gestion des factures en PDF}

	\begin{description}
		\item[Partie 1 :] Impressions des facture en PDF
			\begin{itemize}
				\item Programme tiers : \bsc{PDFCreator}\sautligne

				\item Archivage automatique
				\item Chemin paramétrable
			\end{itemize}~

		\item[Partie 2 : ] Impression automatique
			\begin{itemize}
				\item Lancement régulier\sautligne

				\item Liste factures
				\item Impression factures manquantes
			\end{itemize}
	\end{description}
\end{frame}

% \chapter{Gestion commerciale d'un entrepôt}
\section{Présentation}
Malgré toutes ses fonctionnalités, \integrale{} ne dispose pas de gestion commerciale. En effet, la plupart des entreprises préfèrent se tourner vers un logiciel  tiers comme \bsc{Ciel} pour gérer cet aspect de leur entreprise. Les deux grands problèmes des logiciels de comptabilité, c'est leur prix et le complexité.

Une petite entreprise qui gère seulement un entrepôt n'a certaines pas besoins de l'ensemble des fonctionnalités proposées par \bsc{Ciel} et ses concurrents. Le prix allant de paire avec la complexité du logiciel et sa capacité à couvrir un maximum de chose, ils sont généralement hors de prix pour les petites sociétés. Elles doivent alors faire leur gestion elle-même avec tous les problèmes et risques que cela occasionne.

~

C'est pour offrir un alternative à ce dilemme que j'ai eu ce troisième projet. Le but était d'offrir à ceux qui veulent, un module supplémentaire permettant de faire une gestion commerciale très simplifié des entrepôts. Un des avantages serait l'intégration de ce module, qui aurait des liens directs avec le module de gestion d'entrepôts.

~

Ce projet ne s'est pas fait d'un seul bloc. Il était et est toujours considéré comme mineur, les autres tâches ayant toujours la priorité. J'ai donc commencé le projet au milieu du mois de février 2011 jusqu'à la fin du mois d'avril 2011. Mais il a été régulièrement interrompu par d'autres tâches mineures.

\section{Étude avec UML}
Dans le cas de ce projet, il s'agissait de commencer au tout début. Ma première tâche à donc été d'étudier le projet. Grâce aux cours reçu lors de cette année, mon choix s'est rapidement porté sur l'UML\footnote{UML : \emph{Unified Modeling Language}, un \og langage \fg{} permettant de modéliser les différents aspects d'une tâche, d'un objet. Il est très utilisé en informatique pour modéliser les programmes avant leur développement proprement dit.}.

~

Ce module module ne devant être qu'une version très simplifié d'une \og réelle \fg{} gestion commerciale, je me suis restreint à la gestions des clients / fournisseurs et les deux grands échanges que l'on est amené à partager avec eux : les commandes et les factures.

~

Lors de cette étude, j'ai rapidement remarqué l'étrange symétrie qui existe entre les clients et les fournisseurs d'une entreprise. Même si ils ne s'agit pas de la même chose, même d'un point de vue informatique, les choses fondamentales à retenir sont presque les mêmes. Les échanges, même s'il leur sens est opposé, sont identiques (presque les mêmes caractéristiques).

Afin de simplifier mon étude, j'ai décidé de me concentrer en premier sur une seule moitié de l'étude : celle en rapports avec les clients. Ensuite, dupliquer cette partie en y apportant quelques modifications donnerons l'étude des fournisseurs.

\subsubsection{Réutilisation d'éléments existants}
Les clients et les fournisseurs de ce module ne sont pas les seules \og personnes \fg{} à être décrite dans \integrale. De même, les articles auxquels font références les commandes et les factures existent déjà dans le gestion d'entrepôts.

~

Afin de limiter la duplication inutile des données, j'ai réutilisé des éléments existant de la base de données. Dans certains cas, comme pour les articles, j'ai néanmoins dû prévoir une légèrement intervention sur la structure de ces tables\footnote{Les base de données regroupent leurs données dans des \emph{tables}.}, généralement pour y adjoindre de nouvelles données (comme poiur la table \og article \fg, par exemple).

\section{Réalisation du module}
J'ai commencé par agrandir la base de données pour y ajouter les tables nécessaires. La structure de celle-ci ayant été en grande partie faite lors de l'étude, ça ne m'a pas pris longtemps.

~

Pour débuter le module lui-même, j'ai créé un nouveau formulaire présentant les données de la table au centre de celui-ci : les commandes. Ce formulaire est un petit peu particulier : il est constitué de deux parties. La première est réellement relative aux données de la commande comme la date, le client associé, etc. La seconde partie, quand à elle, est la liste des articles présent dans la commande. Cette partie est écrite dans un second formulaire qui est ensuite intégré au premier.

~

Comme pour tous les formulaires, leur réalisation s'est déroulée en trois temps :
\begin{enumerate}
	\item Création de l'interface visuelle, c'est à dire tout ce qui visible à l'utilisateur.
	\item Création de l'aspect fonctionnel : comment le formulaire réagit aux actions d'un utilisateur (clic sur un bouton, etc.).
	\item Test du formulaire.
\end{enumerate}

~

Suite à cela, j'ai crée deux autre formulaires. Le premier liste les commandes existantes, soit dans leur totalité, soit selon un filtre introduit par le second formulaire. Quand au deuxième, il présente les différents champs qui peuvent être filtré pour trouver une ou plusieurs commandes en particulier.

Cette paire est présente à de nombreuse reprise dans \integrale, je n'ai donc pas mis longtemps à les faire.

~

Après les commandes, je suis passé aux factures. J'ai employé exactement le même principe :
\begin{itemize}
	\item un formulaire détaillant le contenu d'une facture ;
	\item une paire de formulaire pour la recherche.
\end{itemize}

~

Et pour finir, je suis passés aux tests avec de fausses données : création, consultation / modification et suppression de facture.

%\subsubsection{}

\section{Ce qu'il reste à faire}
bla bla bla

% \chapter{Tâches annexes}
\section{Ajustement automatique de la taille}
\subsection{Présentation}
bla bla bla

\section{Système de traduction des fenêtres}
\subsection{Présentation}
bla bla bla

\section{Écriture linéaire d'un nombre}
\subsection{Présentation}
bla bla bla




% Calendrier => annexes
%
% plan prévu :
% - projet tuteuré
% - factures pdf
% 	- impression
% 	- programme génération automatique
% - gestion commerciale
%
% - tâches annexes
% 	- redimensionnement automatique
% 	- traduction
% 	- conversion littérale (int => nombre écrit)

	\part{Synthèse : de mai 2011 à septembre 2011}

\chapter{Introduction}
bla bla bla
\chapter{Tampon-FLASH}
bla bla bla
\subsection{Vente en Plus}

\begin{frame}
	\frametitle{Réécriture du module}

	\begin{description}
		\item[Description :] Administration pure
			\begin{itemize}
				\item Gestion contenu\sautligne

				\item Module recherche
				\item Import / Export documents
			\end{itemize}~

		\item[État : ] Pas démarré
			\begin{itemize}
				\item Délais courts
				\item Projet \og standard \fg
			\end{itemize}
	\end{description}
\end{frame}


	\section{Bilans}

\subsection{Point sur les objectifs}

\begin{frame}
	\frametitle{Point sur les objectifs annuels}

	\begin{itemize}
		\item \solulog{} :
			\begin{itemize}
				\item Complet à 100\%\sautligne

				\item Gestion commerciale
					\begin{itemize}
						\item Réalisé à 75\%
						\item Facile à terminer
					\end{itemize}
			\end{itemize}~

		\item \fidit{} :
			\begin{itemize}
				\item Tampon-FLASH terminé, mais :
					\begin{itemize}
						\item Développement trop long
						\item \og réinvente la roue \fg
					\end{itemize}~

				\item Début de Vente en Plus
			\end{itemize}
	\end{itemize}
\end{frame}

\subsection{Bilan général}

\begin{frame}
	\frametitle{Bilan général}

	\begin{description}
		\item[Contenu :] Très instructif
			\begin{itemize}
				\item Nouveaux langages
				\item Nouvelles techniques\sautligne

				\item Gestion de projet \og pratique \fg
			\end{itemize}~

		\item[Après la licence :] Offre d'emploi
			\begin{itemize}
				\item Employé par \fidit
				\item => Réalisation sites web\sautligne

				\item Missions occasionnelles chez \solulog
			\end{itemize}
	\end{description}
\end{frame}

\subsection{Questions}

\begin{frame}
	\frametitle{Questions}
	
	\begin{center}\begin{Huge}
		Merci de votre attention
		
		~
		
		Avez-vous des questions ?
	\end{Huge}\end{center}
\end{frame}



	\chapter*{Conclusion}
\addcontentsline{toc}{chapter}{Conclusion}
C'est ici que s'achève mon rapport de projet tuteuré mené pour \solulog. Je pense avoir tout dit dessus.

~

Mes précédentes expériences professionnelles dans le domaine de l'informatique se sont déroulées au sein d'un laboratoire de recherche, ce projet tuteuré constitue donc ma première véritable expérience informatique dans le monde du travail.

~

Pour conclure, je suis ravi d'avoir pu mener un tel projet jusqu'à son terme. C'est la première fois que je travaille presque seul sur un projet aussi vaste. Cela n'a pas été facile, mais je suis parvenu au bout.

De plus, j'ai le gratifiant sentiment d'avoir participé activement à un projet utile et essentiel pour l'entreprise.

~

Enfin, je souhaite une fois de plus remercier les acteurs de \solulog{} pour le rôle indiscutable qu'ils ont joué dans mon intégration dans l'équipe de l'entreprise, et au monde du travail d'une manière plus générale.

	\appendix
	\part{Annexes}

\chapter{Calendrier annuel}
Voici un calendrier qui détaille les tâches que j'ai accompli entre le 25 septembre 2010 et le 14 septembre 2011.

\begin{figure}[h!]
	\begin{center}
		\includegraphics[scale=1]{Contenu/Annexes/Images/Calendrier.png}
	\end{center}

	\caption{Calendrier des tâches effectuée pendant mon alternance}
	\label{calendrier}
\end{figure}

\subsection{Tampon-FLASH}

\begin{frame}
	\frametitle{Site Tampon-FLASH}

	\begin{description}
		\item[Description :] Site de vente en ligne de tampons
			\begin{itemize}
				\item Prévisualisation directe
				\item Paiement via PayPal
				\item Module administration
			\end{itemize}~

		\item[Objectifs : ] Découvertes
			\begin{itemize}
				\item \fidit
				\item Gestion de projets \og réelle \fg\sautligne

				\item Langages web
				\item Techniques de programmation
			\end{itemize}
	\end{description}
\end{frame}


\end{document}
