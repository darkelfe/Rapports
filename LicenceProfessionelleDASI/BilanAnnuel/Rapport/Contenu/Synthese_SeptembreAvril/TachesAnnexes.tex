\chapter{Tâches annexes}
Avant d'aborder la seconde partie de mon année, je vais parler brièvement des courts projets que j'ai également mené en même temps que les projets détaillés précédemment.

\section{Ajustement automatique de la taille}
\subsection{Présentation}
Un des désavantages de réaliser nos interfaces avec \bsc{Access} 2003, c'est que celui-ci ne sait pas s'adapter automatiquement à la taille de la fenêtre. C'est-à-dire que quelle que soit la taille de votre écran, l'ensemble a toujours la même taille.

De nos jours, les écrans sont de plus en plus grand. Mais il existe encore des écrans plus petits. Pour que \integrale{} puisse s'adapter à tous les écrans, ses formulaires ont été réalisées pour de petits écrans. Le problème survient lorsqu'on lance le programme avec un grand écran : la fenêtre affichée n'occupe qu'un fragment de l'écran. De plus, les objets présents dedans sont assez petits, ce qui les rends difficile d'accès.

~

Afin de résoudre ce problème, \solulog{} m'a chargé de créer un système capable de redimensionner automatiquement le contenu d'un formulaire.

\subsection{Réalisation}
Avant même de réaliser l'agrandissement des formulaires, j'ai cherché un moyen d'obtenir le facteur d'agrandissement dont j'avais besoin.

~

La première chose à laquelle j'ai pensé, c'est récupérer les dimensions de l'écran actuel et de calculer le facteur par rapport à la taille initiale des fenêtres. Le problème, c'est que le \vb{} est trop ancien pour disposer de fonctions permettant de connaître les caractéristiques de l'écran connecté. J'ai continué à chercher et me suis finalement rendu à l'évidence : il n'existait qu'un seul moyen pour m'en sortir. Heureusement, il s'agit d'une solution très simple : l'utilisateur renseigne le facteur \og à la main \fg.

Par chance, la plupart des d'utilisateurs travaillent sur un poste fixe, toujours avec le même écran. Avec l'aide de mon supérieur, j'ai ajouté un réglage permettant à chaque utilisateur de régler ce facteur.

~

Ensuite, il fallait passer au redimensionnement du formulaire lui-même. De ce côté-là, le \vb{} est bien fait : il dispose de nombreuses fonctions permettant de parcourir l'ensemble des éléments d'un formulaire.

Le principe est simple : parcourir chaque éléments du formulaire, et pour chacun, multiplier ses dimensions par le facteur. Il faut également appliquer le facteur à la position de l'objet : évite que tous les éléments se retrouvent tous les uns sur les autres. Ensuite, si l'élément en question dispose d'un texte (de base ou saisi par un utilisateur), je multiplie la taille de sa police de caractère.

~

Globalement, tout s'est bien passé à l'exception des onglets. Ceux-ci ont un fonctionnement un peu différent et j'ai mis un long moment à identifier et corriger le problème.

\section{Système de traduction des fenêtres}
\subsection{Présentation}
Pour le moment, \integrale{} est un produit entièrement français, il est donc écrit en français. Mais \solulog{} souhaiterai le vendre à l'étranger, ou le mettre en place dans des filières étrangères de nos clients. Le problème qui apparaît immédiatement est celui de la langue. Comme pour la majorité des programmes, il est impossible d'écrire une version du programme par langue. Il fallait donc un système multi-langue, dont la mise en place m'a été confié.

~

Mon supérieur avait déjà une ébauche d'idée sur le sujet : les traductions des textes seraient stockées dans un simple fichier au format INI\footnote{Il s'agit d'un format de fichier simple et connu, qui associe une valeur à une clé.}.

\subsection{Réalisation}
Le chemin pour faire cela était donc tout tracé. Me basant sur mon programme d'ajustement automatique des tailles, je parcours finalement l'ensemble des éléments du formulaire et s'il dispose d'un texte, je recherche celui-ci dans le fichier INI associé à la langue. Si je lui trouve une valeur associée, je la récupère pour la mettre à la place du texte existant.

~

La seule difficulté, en théorie, était de lire les fichiers INI. Mais le \vb{} dispose de nombreuses fonctions permettant d'accéder à un fichier de ce genre. Grâce à un autre exemple présent dans \integrale, j'ai facilement pu comprendre comment lire un fichier INI et mettre moi-même en pratique ces fonctions.

\section{Écriture linéaire d'un nombre}
\subsection{Présentation}
Ce mini-projet c'est déroulé pendant que je travaillais sur le module de gestion commerciale détaillé au chapitre \ref{gestion_commerciale}.

~

Mon maître d'apprentissage en entreprise était à la recherche d'une fonction qui lui permette d'obtenir l'écriture linéaire d'un nombre. Par exemple, transformer \og 142 \fg{} en \og cent quarante-deux \fg.

Je me suis donc lancé dans l'étude d'une fonction de ce genre. Il s'est avéré qu'il n'existe pas, en \vb, de fonction permettant cela. J'ai donc décidé  de l'écrire moi-même.

\subsection{Réalisation}
Dans un premier temps, j'ai cherché un site récapitulant l'ensemble des règles de français conditionnant l'écriture linéaire d'un nombre (qui prend des \og s \fg{} et quand, etc.). Grâce à cela, j'ai écrit une première fonction qui découpe le nombre obtenu par groupes de trois chiffres. Ensuite, je convertis chacun de ces groupes en son expression linéaire. Pour terminer le travail, il ne reste plus qu'à regrouper les expressions avec les séparateurs correspondant (milliers, millions, etc.).
