\chapter{Extraction des données}
\section{Architecture de \integrale}
\integrale{} est un programme qui a presque une architecture orientée \emph{Modèle - Vue - Contrôleur} (aussi appelée MVC).

~

Les données sont dans une base de données externe à Microsoft Access et donc au programme. L'accès aux données se fait par le biais de ADODB, une sur-couche au SQL.

Access regroupe la vue (l'apparence) et le contrôleur (les algorithmes) dans un même projet, mais il y a néanmoins une séparation possible.

~

À l'origine, la vue est restreinte à des objets de type \emph{formulaire}. Certains éléments nécessitent quand même du code pour fonctionner : remplissage automatique en fonction d'autres champs, verrouillage selon certaines conditions, etc.

Les algorithmes, eux, disposent de \emph{modules} pour leur implémentation.

~

Pourtant, on trouve régulièrement des algorithmes directement dans le code d'une vue. Il y a une raison à cela : la majorité des formulaires utilisés en \vb{} sont dits \emph{basés}, c'est-à-dire que les données saisies dans un formulaire sont automatiquement synchronisées avec une table spécifiée de la base de données.

Il existe deux moyens d'accéder aux données saisies par l'utilisateur : interroger directement la base de données, ou faire appel aux contrôles du formulaire. Bien évidemment, il est plus aisé d'utiliser la seconde méthode, les données étant directement accessibles à travers une variable, sans nécessiter de code SQL.

Malheureusement, l'accès à ces variables n'est possible qu'au sein du code associé aux vues. Cela explique pourquoi nous trouvons beaucoup d'algorithmes directement dans les vues.

~

Ce cas de figure est peut-être assez répandu chez \solulog, mais quelques fois néanmoins les algorithmes sont au bon endroit : dans un module. Vu les dimensions du code que j'avais à réécrire, mon propre code est venu se placer dans un de ces modules (c'est même un projet Access à part entière qui lui a été dédié) pour ne pas surcharger le code du formulaire de saisie des données.


\section{Structure des données - Forme numéro 1}
Quand un code se place dans un \emph{module}, la politique de \solulog{} est de lire les données essentielles en les stockant dans des variables, puis de lancer le traitement.

~

La première chose à faire est de créer les variables qui permettront de stocker les valeurs issues de la base de donnée. Dans mon cas, il y a un nombre important de  variables (des images du formulaire de saisie sont disponibles dans l'annexe \ref{screenshots}), le plus facile est donc de les regrouper dans une classe. Mais mon maître d'apprentissage m'a indiqué qu'il ne souhaitait pas voir de classes dans \integrale, et j'ai donc opté pour de simples structures, dédiées uniquement au stockage, sans procédures ou fonctions associées.

~

Le formulaire étant découpé en plusieurs parties (en-tête, complément, colisage et credoc\footnote{Credoc : abrégé de \og crédit documentaire \fg.}), j'ai reproduit cette configuration avec les structures. Les données étaient organisées comme suit : une grande structure, appelée BL, regroupait l'ensemble des données. Cela me permettait de ne manipuler qu'une seule grande variable. A l'intérieur, j'ai placé les quatre parties avec leurs données. En plus de cela, j'ai créé deux structures supplémentaires pour le colisage. La première afin de représenter une marchandise, avec une section (structure) consacrée aux informations sur les produits dangereux ; la seconde correspond à un conteneur qui peut avoir plusieurs marchandises, ce qu'on appelle un \emph{tableau} de marchandises.


\section{Lecture des données}
Après la création des structures, je suis passé à l'étape de lecture de la base de données. Pour cela j'ai dû me familiariser avec ADODB, la librairie Microsoft utilisée pour se connecter à une base de données quelconque ; ainsi que les tables contenant les informations désirées.

~

Fidèle aux parties du formulaire et plus précisément aux structures utilisées pour le représenter, j'ai créé au minimum une fonction par structure. En plus de cela, j'ai été amené à créer d'autres fonctions de lecture pour récupérer, par exemple, le nombre de conteneurs et de marchandises par conteneur.

~

Mon maître de stage est ensuite venu faire le point. C'est à ce moment-là qu'il m'a indiqué que les structures de données ne lui semblaient pas mieux que des classes. N'étant pas d'accord avec lui, nous avons beaucoup discuté.


\section{Structures des données - Forme numéro 2}
La principale difficulté posée par la suppression des structures était d'ordre pratique. En effet, l'abandon de celles utilisées pour les tableaux, c'est-à-dire \emph{conteneur} et \emph{marchandise}, promettait d'être problématique.

~

Il serait assez simple de transformer le contenu des autres structures en variables globales au fichier source actuel, mais dans le cas des tableaux, cela devenait plus complexe. Je n'ai vu que deux solutions :
\begin{itemize}
	\item conserver les structures ;
	\item créer un tableau par variable de chaque structure.
\end{itemize}

~

En choisissant la seconde possibilité, les variables de la structure des marchandises vont se transformer en tableaux à deux dimensions. La première pour le numéro de conteneur, la seconde pour le numéro de la marchandise. Et compte tenu du nombre de variables présentes dans ces deux structures, il aurait été nécessaire de créer deux ou trois \emph{dizaines} de tableaux.

Il s'agit d'une option parfaitement viable, c'est celle utilisée pour le module PHP, mais elle est beaucoup moins esthétique et pratique que l'utilisation d'une structure.

~

Après discussion avec M \bsc{Perisse}, nous sommes tombés d'accord pour préserver les structures pour les conteneurs et les marchandises, mais les autres devaient être détruites et remplacées par des variables globales.

Après l'avoir fait, j'ai dû adapter les fonctions de lecture. Après cela, presque deux semaines s'étaient déjà écoulées depuis le début du projet.
\vfill