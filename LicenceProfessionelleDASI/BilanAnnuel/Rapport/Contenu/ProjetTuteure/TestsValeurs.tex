\chapter{Tests des valeurs}
Une fois les valeurs extraites de la base de données, j'ai pu continuer avec la seconde étape, l'une des plus importantes : vérifier les valeurs récupérées. Pour le module \pireus, il y a deux séries de tests qui sont implémentées.

~

Le première, en \vb, est la plus simple, elle vérifie seulement que toutes les données nécessaires ont été saisies. Elle s'assure également de la cohérence entres les valeurs et leur type : un champ attendant une date doit donner une date, etc.

~

La seconde série de tests est présente du côté PHP du module. Elle est chargée de vérifier que les champs qui ne sont requis que sous certaines conditions soient présents si nécessaire. Par exemple : la tare (poids à vide) du conteneur est obligatoire uniquement si le port de destination est le Brésil. Le test se charge également de vérifier diverses choses telles que la longueur maximale autorisée pour certaines chaînes de caractères (exemple, une ligne d'adresse ne peut pas dépasser trente-cinq caractères).

De plus, les données reçues par \pireus{} pouvaient en théorie provenir d'autres source que \integrale. La seconde série de tests doit donc également faire les test de la première série, au cas où ceux-ci n'aurait pas été fait par l'appelant.

~

Dans un souci de simplicité, j'ai voulu réunir tous les tests dans une fonction unique. Souhaitant faire disparaître la séparation existant entre les deux séries de tests de \pireus, j'ai créé un fichier qui listait l'ensemble des ces tests. J'ai ainsi pu éliminer de nombreux doublons et regrouper les tests par entité (d'abord ceux du client, puis ceux de la banque, etc.).

Ensuite je les ai simplement recopiés, les adaptant au \vb. C'est à cette occasion que j'ai expérimenté pour la première fois l'utilisation des expressions régulières\footnote{Expression régulière : pseudo-langage permettant de décrire la structure qu'une chaîne de caractère doit respecter.} (en \vb) dans la vérification du nom des conteneurs : celui-ci doit commencer par quatre lettres, suivies de sept chiffres.

~

Les tests ayant déjà été écrits, il m'a suffit de les reprendre et de les convertir. J'ai passé environ deux jours et demi sur le sujet (listage des fonctions compris).