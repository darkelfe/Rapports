\chapter{Gestion commerciale d'un entrepôt}
\section{Présentation}
Malgré toutes ses fonctionnalités, \integrale{} ne dispose pas de gestion commerciale. En effet, la plupart des entreprises préfèrent se tourner vers un logiciel  tiers comme \bsc{Ciel} pour gérer cet aspect de leur entreprise. Les deux grands problèmes des logiciels de comptabilité, c'est leur prix et le complexité.

Une petite entreprise qui gère seulement un entrepôt n'a certaines pas besoins de l'ensemble des fonctionnalités proposées par \bsc{Ciel} et ses concurrents. Le prix allant de paire avec la complexité du logiciel et sa capacité à couvrir un maximum de chose, ils sont généralement hors de prix pour les petites sociétés. Elles doivent alors faire leur gestion elle-même avec tous les problèmes et risques que cela occasionne.

~

C'est pour offrir un alternative à ce dilemme que j'ai eu ce troisième projet. Le but était d'offrir à ceux qui veulent, un module supplémentaire permettant de faire une gestion commerciale très simplifié des entrepôts. Un des avantages serait l'intégration de ce module, qui aurait des liens directs avec le module de gestion d'entrepôts.

\section{Étude avec UML}
Dans le cas de ce projet, il s'agissait de commencer au tout début. Ma première tâche à donc été d'étudier le projet. Grâce aux cours reçu lors de cette année, mon choix s'est rapidement porté sur l'UML\footnote{UML : \emph{Unified Modeling Language}, un \og langage \fg{} permettant de modéliser les différents aspects d'une tâche, d'un objet. Il est très utilisé en informatique pour modéliser les programmes avant leur développement proprement dit.}.

~

Ce module module ne devant être qu'une version très simplifié d'une \og réelle \fg{} gestion commerciale, je me suis restreint à la gestions des clients / fournisseurs et les deux grands échanges que l'on est amené à partager avec eux : les commandes et les factures.

~

Lors de cette étude, j'ai rapidement remarqué l'étrange symétrie qui existe entre les clients et les fournisseurs d'une entreprise. Même si ils ne s'agit pas de la même chose, même d'un point de vue informatique, les choses fondamentales à retenir sont presque les mêmes. Les échanges, même s'il leur sens est opposé, sont identiques (presque les mêmes caractéristiques).

\section{Réalisation du module}
bla bla bla

\section{Ce qu'il reste à faire}
bla bla bla
