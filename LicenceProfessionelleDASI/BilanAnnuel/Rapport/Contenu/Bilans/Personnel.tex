\chapter{Bilan personnel}
Cette année, en termes de compétences, m'a apporté bien plus que je n'aurais pu l'imaginer.

\section{Solulog}
Parlons du \vb. Il s'agit d'un premier langage que j'ai appris. Malgré ma grande pratique de celui-ci par la passé, les années ont fini par effacer un partie de lui de ma mémoire. C'est avec un immense plaisir que j'ai repris l'étude et la mise en œuvre de ce langage cette année.

~

Grâce au \vb{} et surtout à \integrale, j'ai eu l'occasion de côtoyer d'autres technologie informatique. Le meilleur exemple qui me vient, c'est le PHP, tant par le biais de \pireus{} que dans la réalisation de sites web avec \fidit.

J'ai également découvert les \emph{Boat Landing} et la norme IFTMIN D99B. Malgré la complexité de celle-ci, j'ai énormément apprécié d'avoir pu mettre en place un tel système et de le voir mis en service chez plusieurs clients.

~

En ce qui concerne l'impression en PDF, l'utilisation d'une imprimante virtuelle comme programme tiers à été extrêmement révélateur pour moi. Cet aspect du projet ma donné l'occasion d'écrire un module complet sur un domaine dans lequel je suis le seul expert de \solulog. Il est satisfaisant et également responsabilisant de savoir que tout le reste de l'entreprise repose entièrement et uniquement sur vous pour cette technologie en particulier.

~

Le module de gestion commerciale c'est également montré utile, notamment dans le cas des entrepôts. Le module, tel qu'il est construit m'a obligé à penser sur deux niveaux en même temps. D'un premier côté, il doit être vu comme un programme à part entière. Il dispose de ses propres réglages et chaque donnée se recoupe. J'ai vraiment eu l'impression d'écrire un programme totalement nouveau.

Le second côté porte sur les liens que doit avoir ce programme avec le module de gestion d'entrepôts. Ce dernier ne doit pas être modifié de n'importe quelle manière. Les liens vers la gestion commerciale doivent venir s'y intégrer sans perturber son fonctionnement si le module en question est absent. Il s'agit d'améliorer un code existant sans le modifier de manière trop radicale.

~

Ces deux projets, par leurs connexions avec les factures m'ont poussé à m'intéresser à la manière dont \integrale{} imprime ses factures. Cela m'a fait découvrir les \og états \fg{} d'\bsc{Access} et m'a plongé un peu plus au cœur de ce logiciel.

\section{Fidit}
Bien que cet aspect de mon année soit arrivé un peu à l'improviste, cela m'a donné l'occasion de découvrir la gestion de projet sous un angle inédit. Après avoir vu Jean-Michel \bsc{Colas} la pratiquer chaque jour se révèle une expérience très différente de l'étude théorique vue en cours.

~

Cela m'a également fait plonger à l'intérieur des techniques de développement de \fidit. J'y ai découvert de nombreuses technologies (système de paiement en ligne, le langage javascript) ainsi que des méthodes de programmation innovantes.
