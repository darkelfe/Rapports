\chapter{Activité professionnelle de \fidit}
\fidit{} est spécialisé dans la création et la maintenance de solutions web. Je vais présenter deux d'entre-elles.

\section{Les sites web}
Que ça ne fasse que présenter l'entreprise ou que ça aille jusqu'à intégrer un système de vente en ligne, les sites web sont désormais une force commerciale importante. Ce qui explique que de plus en plus d'entreprise souhaitent avoir le leur, ne serait-ce que pour présenter leurs activités et un moyen de les contacter (adresse postale, mail, etc.).

~

Même s'il peut paraître aisé de créer un site web, obtenir quelque chose de \emph{professionnel} n'est pas si facile. La plupart des entreprises ne disposent pas ou ne veulent pas mettre en place les ressources nécessaire au développement d'un site web.

C'est pourquoi une majorité d'entre elles font appel à des sociétés spécialisée dans la création et le maintient de site web, comme \fidit.

~

\fidit{} est donc là pour créer des sites web de type professionnel qui correspondent aux besoins et à la volonté de ses clients.

\section{Les autres développements}
En plus des sites web, \fidit{} développe et maintient plusieurs applications et solutions en rapport avec internet et les réseaux.

~

Par exemple, le logiciel \emph{Gibus}. Il s'agit d'un programme de gestion de projets, utilisable entièrement par le biais de votre navigateur internet préféré.

Nous pouvons également citer \emph{Axeone}, une solution réseau qui offre un ensemble de programmes pré-installé et configurés, qui sont facilement mis en place directement sur un serveur.
