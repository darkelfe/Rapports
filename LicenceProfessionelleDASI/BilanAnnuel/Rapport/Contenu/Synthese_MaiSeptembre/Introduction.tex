\chapter{Introduction}
Lorsque je suis arrivé à la fin du mois de mars 2011, mon supérieur m'a demandé si j'accepterais d'aller réaliser un ou plusieurs sites web au sein de \fidit{}, mais pour le compte de \solulog. Cette occasion m'offrait plusieurs avantages : changer de langage de programmation, connaître le fonctionnement interne de \fidit, etc.

~

Après avec longuement réfléchi, j'ai finalement accepté l'offre. Ce changement se ferait lorsque j'aurais terminé la partie client de la gestion commerciale (chapitre \ref{gestion_commerciale}). C'est finalement tombé au tout début du mois de mai 2011.

~

Ce projet, nommé \og Tampon-FLASH \fg, m'a finalement occupé jusqu'au mois de septembre de cette même année. Il est également présent dans le calendrier visible dans l'annexe \ref{calendrier} à la page \pageref{calendrier}.

\subsubsection{Plan de la partie}
Dans cette partie, je vais détailler \emph{deux} projets. Le premier est bien évidemment Tampon-FLASH. Le second projet s'appelle \og Vente en Plus \fg. Il s'agit d'un autre site web que je n'ai pas encore commencé (à l'heure de la soutenance).
