\chapter{Bilan personnel}
Ce qui m'a, sans conteste, le plus apporté en termes de compétences, c'est le code de \pireus. Comme je l'ai déjà dit, en démarrant l'année, je connaissait déjà le PHP, mais seulement de manière \og basique \fg. L'étude du code m'a appris de nombreuses techniques sur le PHP et, suite aux cours d'architecture n-tier, sur les web services et SOAP.

En effet, \pireus{} a été pour moi un exemple réel et fonctionnel d'un web service, sans aucun point commun avec les \og cas d'école \fg.

~

Avant de travailler pour \solulog, j'avais beaucoup étudié et utilisé le \vb. Mais après plusieurs années sans employer ce savoir, il avait finit par s'estomper de ma mémoire. La réécriture du module dans \integrale{} m'a permis de rafraîchir ces connaissances.

~

Grâce aux BL, j'ai eu le plaisir de découvrir les EDI et... la malchance de travailler avec l'un deux. J'en parle ainsi car il s'agit d'un EDI particulièrement difficile à mettre en place. Une complexité que j'ai toutefois pu pleinement apprécier, parce qu'elle m'a forcé à me surpasser.

Travailler sur les BL m'a également fait découvrir un métier complexe, que je ne connaissais pas : l'organisation du transport international. Je n'avais jamais réfléchit aux moyens qui devaient être mis en place pour cela, et je dois admettre qu'il faut bien plus de choses que je n'aurais pu l'imaginer.

~

La réécriture du module a été une chance pour moi, car il s'agissait de mener un projet presque d'un bout à l'autre. Même si une grande partie des algorithmes était déjà en place pour le module PHP, il a fallu que je les étudie, que je comprenne leur fonctionnement et leur raison d'être. Malgré leur complexité, j'ai pris beaucoup de plaisir à les explorer et à les adapter au \vb.

J'ai également beaucoup apprécié de pouvoir suivre l'évolution d'un programme étape après étape et d'y avoir participé à chaque instant. De plus, avec du recul, observer le programme que l'on a réussi à créer apporte une grande satisfaction.

~

Il est certain que recréer le module aurait été bien plus simple avec l'aide physique d'Alexandre. Mais son absence m'a obligé à me surpasser pour comprendre et mettre en \oe{uvre} les principes sous-jacents. J'en suis d'autant plus fier que j'ai finalement réussi avec très peu d'aide.