\chapter{Vocabulaire propre à ce document}
Dans le but de simplifier la communication et éviter les ambiguïtés, je vais préciser quel sens je donne à certains termes dans ce document.

\section{Termes généraux}
Un \emph{client} fait référence à un client de \integrale, et par extension de \solulog. Il s'agit donc d'une entreprise organisant le transport international (la partie logistique ne sera pas abordée dans ce projet tuteuré).

~

Un \emph{utilisateur} représente un employé de la société cliente utilisant couramment \integrale.

~

Un \emph{client final} est une entreprise qui souhaite transporter des marchandises.

~

Une \emph{recette} fait référence au test d'un programme directement chez les clients.

\section{Termes relatifs au \vb}
Un \emph{formulaire} est la représentation dans le code d'une fenêtre de l'application.

~

Un \emph{contrôle} est un élément constituant un formulaire, tel un onglet, un champ texte ou encore un bouton.

~

Un \emph{module} ou \emph{fichier source} est un regroupement de fonctions / procédures et de déclarations de variables (optionnel).

~

Un \emph{module de classe} est un module qui contient la définition d'une classe (Programmation Orientée Objet).