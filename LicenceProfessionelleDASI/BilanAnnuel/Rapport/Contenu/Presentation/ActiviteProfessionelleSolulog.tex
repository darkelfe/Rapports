\chapter{Activité professionnelle de \solulog}
\solulog{} développe une série de progiciels nommée \integrale. Elle s'organise autour de deux grands domaines : l'organisation du transport international et la logistique.

\section{L'organisation du transport}
\subsection{Principe}
Quand une entreprise souhaite transporter des marchandises d'un endroit à un autre, elle a deux moyens à sa disposition :
\begin{itemize}
	\item organiser le transport elle-même, ce qui est très complexe et difficile à mettre en \oe uvre. C'est pourquoi seules les très grandes entreprises ont recours à ce moyen ;
	\item demander à une entreprise spécialisée de le faire pour eux : des sociétés d'organisation du transport.
\end{itemize}

~

Il existe deux catégories d'entreprises qui organisent le transport. Les premières sont spécialisées dans le transport national, c'est-à-dire des marchandises qui se déplacent seulement au sein d'un même pays. Par exemple, transporter des citrons du midi de la France à Paris est un transport dit national.

La seconde catégorie correspond aux entreprises qui préfèrent le transport \emph{international}. Elles s'occupent donc de l'import et l'export de marchandises (chaussures, bananes, etc.)

Il existe des entreprises qui offrent les deux possibilités, mais elles sont rares.


\subsection{Les sociétés d'organisation}
Sans vouloir minimiser le travail requis pour le transport au sein d'un même pays, l'organisation au niveau international est extrêmement complexe. En plus des tâches \og communes \fg {}, comme la réservation d'un moyen de transport, il faut déclarer les marchandises aux douanes, informer la compagnie de transport et diverses autorités si les marchandises sont d'un type considéré comme dangereux (par exemple explosif ou des produits chimiques), etc.

~

L'organisation du transport présente de nombreux aspects qui n'intéressent absolument pas l'entreprise cliente. C'est pourquoi les sociétés qui organisent le transport se veulent très simples \og d'utilisation \fg.
Ainsi, il suffit de leur communiquer le type de marchandises, à transporter d'un point A à un point B. Il est également possible d'imposer quelques restriction, par exemple le moyen de transport : maritime, routier ou aérien.

A partir de ces données, l'entreprise s'organise tout seule, en remplissant les documents nécessaires, réservant un moyen de transport, allant parfois jusqu'à récupérer elle-même les marchandises dans l'entrepôt.

~

En règle générale, il s'agit de sociétés exclusivement de services qui organisent le transport. Elles ne possèdent rien de physique et servent de pont entre les diverses entités impliquées dans le transport.


\subsection{La place de \integrale}
Mais quelles est donc la place du logiciel \integrale{} dans tout cela ?

~

Malgré les experts et l'expérience des entreprises, de nombreuses tâches peuvent être simplifiées : système de facturation, envoi automatique de document, etc.

C'est là que \solulog{} et ses deux concurrents français entrent en jeu. Ils développent un ou plusieurs programmes destinés à automatiser et simplifier au maximum le travail des sociétés organisatrices de transport international.


\section{La logistique}
Il existe de nombreuses entreprises (généralement celles qui souhaitent transporter des marchandises) qui possèdent des entrepôts. Qu'il s'agisse de leur activité principale ou d'un service proposé à leurs clients, gérer un entrepôt est un tâche difficile.

Il est nécessaire d'avoir un état \emph{constant} des stocks, un récapitulatif des ressources qui entrent et qui sortent, mais aussi un système pour organiser le rangement et autres outils.

~

La quantité de données à gérer et à traiter est telle que presque toutes les entreprises (même les petites structures) font appel à un logiciel dédié. Ce domaine étant assez lié à l'organisation du transport (les marchandises à transporter sont stockées dans des entrepôts), \integrale{} est donc, soit comme logiciel indépendant, soit en lien avec le reste, une solution pour gérer les entrepôts.
