\chapter{Bilan professionnel}
Du coté de l'entreprise, je pense avoir parfaitement réussi à mener mes objectifs à bien. En effet, il existe maintenant un module parallèle à \pireus, mais entièrement en \vb. En outre, il dispose désormais de davantage de fonctionnalités que le module originel (NVOCC, etc.)

Ce nouveau module est désormais en place chez plusieurs clients et semble fonctionner pour le mieux. Les objectifs semblent donc être remplis.

~

Ma curiosité naturelle m'a poussé, tout au long de ce projet, à étudier des aspects du \vb{} plus poussés que nécessaire. Certains d'entre eux m'ont semblé être utiles pour \integrale, et je les ai donc proposés à \solulog, qui les a parfois mis en \oe{uvre}.

~

J'ai également pu apporter une certaine expertise en programmation et pour améliorer le code existant. J'ai su identifier des fragments de code récurrents et les factoriser dans des fonctions.

~


Du point de vue de mon orientation professionnelle, ce projet m'a grandement conforté dans l'idée que le développement informatique est une voie qui me plaît. Je ne regrette aucun moment passé à réfléchir au code d'Alexandre ou à résoudre un problème technique.

J'ai pu également me prouver que je ne m'ennuie pas malgré plusieurs semaines consécutives sur un même projet. Chaque étape de la création d'un programme possède une diversité qui lui est propre, provoquant un renouvellement constant de l'intérêt.

De plus, j'ai appris que même en travaillant avec un langage qui \og vieillit \fg, je peux prendre beaucoup de plaisir.