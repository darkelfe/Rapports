\chapter{Environnement de travail}
\section{\solulog}
L'ensemble des données de \integrale{} est stocké dans une base de données Microsoft SQL Server (soit 2005, soit 2008). Le logiciel lui-même doit être installé sur un serveur du client.

~

Dans le cadre de son développement, le code source est enregistré sur un serveur distant de \solulog{} et hébergé par \fidit. Celui-ci utilise le système d'exploitation Microsoft Windows Server 2003.

~

La modification du code source ou la réalisation de tests se fait donc directement sur le serveur en question. Chaque employé de \solulog{} dispose de sa propre session. L'accès se fait par TSE\footnote{TSE : Terminal Server Edition, une méthode pour se connecter à distance à une session \bsc{Windows}.}. De cette manière, toutes les données sont centralisées et sauvegardées quotidiennement.

~

Le programme \integrale{} est développé dans Microsoft Access 2003 (mais fonctionne également avec les versions 2007 et 2010) avec le langage \vb.

~

Access possède lui-même une base de données interne, mais la taille de celle de \integrale{} a obligé \solulog{} à les externaliser vers une base de données spécialisée (ici, Microsoft SQL Server).

Au final, Access ne sert donc que pour les parties interface et fonctionnelle du programme.

\section{\fidit}
Chez \fidit, le travail est réparti entre deux types de serveurs. Les premiers permettent le développement proprement dit des sites web. Ils sont configurés pour le développement web. Les seconds type de serveurs sont là pour les versions finales des sites web. C'est sur ceux-là que les les utilisateurs vont se connecter et utiliser les sites web.

~

Tous les serveurs de \fidit, quelque soit leur type, sont géré de manière identiques : avec \bsc{Linux}\footnote{Comme \bsc{Windows}, \bsc{Linux} est un système d'exploitation.}. Les employés travaillent directement sur le contenu des ces serveurs par le biais de connexions sécurisées dites SSH\footnote{SSH : Secure SHell, une interface d'échanges distant, disponible uniquement en mode texte.}.

La base de donnée est la même pour tous les sites : MySQL.

~

L'édition de fichiers sources constituant les sites web est faite par avec le logiciel VIM. En plus des connexions SSH, il est également possible d'importer directement des fichiers (comme les images) en utilisant une connexion FTP\footnote{FTP : File Transfert Protocol, une méthode permettant le transfert de fichiers entre deux entités distantes.}.

~

Pour finir, les développements de \fidit{} utilisent au total cinq langages de programmations, mais qui constituent \emph{ensemble} la base des sites web. En voici la liste avec une courte description :
\begin{itemize}
	\item Le HTML : permet de décrire le contenu de la page (le titre, les menus, etc.) ;
	\item Le CSS : contient les éléments relatifs à l'aspect de la page (le menu est à gauche, les liens sont en rouge, etc.) ;
	\item Le PHP : il s'agit de la partie \og réflexion \fg du site, celle qui interagit avec la base de donnée ;
	\item Le SQL : permet les échanges avec la base de donnée, il est utilisé via le PHP.
	\item Le javascript : également la partie réflexion, il joue un rôle au niveau du navigateur internet de l'utilisateur.
\end{itemize}
