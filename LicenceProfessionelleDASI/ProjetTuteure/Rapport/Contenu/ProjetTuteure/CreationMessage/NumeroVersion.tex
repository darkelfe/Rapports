\section{La problématique du numéro de version}
\subsection{Présentation}
Chaque message de BL comporte un segment \og BGM+340 \fg. Celui-ci contient trois informations essentielles :
\begin{enumerate}
	\item Le numéro de message (id) ;
	\item Le numéro de version ;
	\item Le type de message : nouveau / existant.
\end{enumerate}

~

Le numéro de message est unique, ce qui permet donc d'identifier un BL sans confusion possible. En moyenne, un client ne crée pas plus de quatre ou cinq messages de BL par jour. Pour l'id, j'ai donc décidé de procéder comme Alexandre : en utilisant la date et heure actuelles. Le numéro de message est donc le résultat de la concaténation de la lettre I (comme ID), de l'année sur deux chiffres, du mois, du jour, de l'heure (mode 24h), des minutes et des secondes. Par exemple, un message créé le 23 juillet 2010 à 11 heures, 12 minutes et 57 secondes portera donc l'id suivant : \og I100723111257 \fg.

~

Malgré la vigilance des personnes chargées de saisir les informations du BL, il arrive de temps en temps que des erreurs se glissent dans les documents (exemple : l'orthographe du nom de la banque, etc.). Ces erreurs, compte tenu de leur passage dans les fonctions de vérification du module, ne faussent pas la structure du message. Si l'opérateur se rend compte de son erreur (c'est généralement le cas), il va vouloir corriger le problème. Malheureusement, cela ne serait pas une bonne idée d'envoyer deux BL différents portant exactement sur les mêmes données. En effet, comment la compagnie maritime pourrait-elle identifier le document valide de celui erroné ? Le temps de traitement étant variable, même les repères temporels ne seraient pas fiables.

~

C'est à ce moment-là qu'INTTRA entre en jeu. Pour résoudre ce problème, INTTRA autorise donc un client à envoyer deux BL portant le même ID. Mais pour pouvoir distinguer le plus récent de l'ancien, le segment BGM inclut un champ \emph{numéro de version}.

Je m'explique : l'ID permet d'indiquer à INTTRA que les deux messages de BL portent sur le même contenu. Et le numéro de version permet de déterminer quel message et le plus à jour. Plus le numéro de version est élevé, plus le message à jour. Et afin de détecter les vrais doublons, INTTRA a placé un troisième champ dans le segment BGM : le type de message. Celui-ci, grâce à sa valeur, permet de dire sans ambiguïté si ce message est un nouveau BL, ou s'il s'agit d'un message de révision. Dans le premier cas, aucun autre message à l'exception des révisions ne doit porter le même ID. Dans le second, il s'agit d'un message de révision, et son ID permet d'identifier le message d'origine.


\subsection{Gestion par \integrale}
Le problème c'est que les utilisateurs ne connaissent pas et ne veulent pas connaître cet aspect. Sans parler des risques causés par un oubli de la case à cocher \og ce document est une révision \fg.

Cette propriété est donc à la charge de \integrale{} : si un utilisateur cherche à expédier un document de BL déjà envoyé, le progiciel doit vérifier si les données ont changé. Si c'est le cas, nous envoyons le BL avec le même ID que le message d'origine, mais avec un numéro version incrémenté de un. Sinon, nous avertissons l'utilisateur qu'un BL identique a déjà été transmis à INTTRA.

~

La seule manière de pouvoir faire cela est de garder une trace de tous les BL envoyés. De plus, de nombreux clients souhaitent pouvoir garder une trace de chaque opération menée, et par qui. Nous avons donc joint ces deux aspects en créant un système d'audit : à chaque fois qu'un utilisateur génère un B/L, l'évènement est enregistré avec diverses valeurs.

Nous y avons intégrés plusieurs champs pour résoudre le problème des versions. Tout d'abord, un id interne à \solulog{} et qui permet de savoir si deux BL portent sur les mêmes données. Si c'est le cas, nous devons déterminer si les données ont été modifiées.

~

La grande question est la suivante : comment faire cela ? Les données de la nouvelle tentative d'envoi sont déjà enregistrées dans la base de données, donc impossible de comparer les champs avec ceux du BL d'origine. La réponse à cette question provient des cours de la licence professionnelle. En effet, savoir si des données sont identiques à celles de départ revient à vérifier leur intégrité. Et c'est une chose que les réseaux informatiques (pour des raisons de sécurité concernant les données transmises) ont beaucoup étudié. Je n'avais donc plus qu'à utiliser leur système.

~

La technique standard consiste à faire une somme de contrôle (souvent nommée \emph{checksum}, on parle aussi de \og hachage \fg) qui transforme un texte en un numéro unique propre au texte (c'est-à-dire que deux textes identiques renverront le même numéro). Il existe de nombreux algorithmes standardisés permettant de le faire : MD5, SHA1, etc. Il ne restait donc plus qu'à le mettre en place dans le module (c'est d'ailleurs la technique employée en PHP par Alexandre).

Mais le \vb{} m'a réservé une petite surprise : il n'existe aucune bibliothèque ou fonction qui permette de faire de hachage ! Bien sur, il est possible de créer sa propre de fonction, mais les algorithmes mis en \oe{uvre} sont très compliqués. Alors, plutôt que de stocker le hachage de la chaîne, j'ai décidé de stocker directement la \emph{chaîne de caractères représentative des données}.

~

L'expression de \og chaîne de caractères représentative des données \fg{} est très jolie, mais n'explique pas en quoi elle consiste. En fait, il s'agit d'une chaine de caractères qui contient les données du formulaire (celles importantes). Je la construis à partir d'un processus dérivé de la \emph{sérialisation} des données. La sérialisation est une opération informatique qui consiste à stocker consécutivement plusieurs données, tout en ayant un moyen de les retrouver plus tard (on parle dans de ce cas de \emph{désérialisation}). Mais dans notre cas, l'extraction des données est sans importance, alors je stocke simplement l'ensemble des valeurs d'un BL (exception faite pour les données sans importance comme l'utilisateur à l'origine du BL) les unes après les autres.

~

C'est cette grande chaîne que je stocke dans la table d'audit. Si celle générée par les données du BL actuel est différente de celle mémorisée dans la table, c'est que quelque chose a changé (peu importe quoi). Dans ce cas, j'utilise deux autre champs de la table d'audit : un qui stocke l'id du message et l'autre son numéro de version. Grâce à eux, je peux transmettre le nouveau message avec le même id que celui d'origine, mais avec un numéro de version plus élevé.
Dans le cas contraire, où les données sont les mêmes, j'affiche un message d'erreur indiquant l'impossibilité d'envoyer deux fois le même message de BL.
~

En accord avec le principe de l'audit (et également dans le cas où le message serait encore modifié), la nouvelle version du message doit être mémorisée sans pour autant écraser le message d'origine. Pour cela, je crée une nouvelle ligne dans la table, avec les informations au message corrigé. Il faut juste faire attention lors de la vérification de l'existence d'un message de BL déjà envoyé : le résultat de la requête SQL doit être trié par ordre décroissant de numéros de version et choisir le premier (afin de faire la comparaison avec la dernière version du message).

La table d'audit stocke aussi plusieurs autres informations, comme l'utilisateur ou le message de BL généré.

~

La mise en place du système m'a pris environ une semaine : la table d'audit existait déjà, il a juste fallu ajouter les champs nécessaire au BL.