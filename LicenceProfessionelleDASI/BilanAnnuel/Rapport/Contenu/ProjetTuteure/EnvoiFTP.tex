\chapter{Envoi par FTP}
Après une phase rigoureuse de tests, j'ai fait la fonction d'envoi par FTP.

~

L'envoi de documents se fait par un dépôt des fichiers de BL sur un FTP d'INTTRA. Chaque client dispose d'un ou plusieurs comptes, les informations d'accès étant réglées dans le paramétrage d'\integrale.

~

La mise en place de l'envoi a été très simple pour deux raisons. La première est que grâce à Microsoft, le \vb{} dispose de fonctions permettant la communication avec des comptes FTP. La seconde raison est qu'en plus de ces fonctions, Alexandre a développé une sur-couche objet via un module de classe. Celle-ci ne sert essentiellement qu'à clarifier les appels : des noms plus simples pour les fonctions et les paramètres.

La fonction de connexion fait cependant exception à la règle, une vérification de la connexion étant effectuée.

~

Il ne me restait plus qu'à utiliser les fonctions pour l'envoi. Rien de compliqué ici : se connecter, envoyer un fichier, vérifier son envoi puis se déconnecter.

~

La mise en place du FTP ne m'a pas pris plus de quelques heures.
\vfill