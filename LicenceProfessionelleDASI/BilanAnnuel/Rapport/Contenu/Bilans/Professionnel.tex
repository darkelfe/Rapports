\chapter{Bilan professionnel}
Cette année m'a aussi donné l'occasion de m'adapter aux entreprises pour lesquelles j'ai directement travaillé. J'ai eu la possibilité de prendre mes marques et me conformer aux règles qui régissent le code.

~

Les membres de \solulog{} et \fidit{} ont également la possibilité de profiter de mon expérience dans les nouvelles technologies directement à l'oral, ou par le biais de leur mise en place dans plusieurs de mes outils.

\section{Solulog}
Pour le compte de \solulog, je pense avoir totalement atteint les objectifs qui m'ont été confiés. J'ai réussi à mener à bout l'ensemble des projets qui étaient sous mon contrôle, à l'exception de la gestion commerciale. Mais celle-ci n'attend que mon retour pour reprendre son développement.

~

J'espère sincèrement avoir pu les aider du mieux de mes capacités.

\section{Fidit}
Du côté de \fidit, les objectifs sont peut être moins bon que ceux mené pour \solulog. Même si le projet Tampon-FLASH était d'une taille conséquente, j'ai passé beaucoup de temps, peut être trop, à le développer. \fidit, tout comme moi-même, sommes conscient que c'est essentiellement une accumulations de plusieurs facteurs comme l'absence d'un cahier des charges ou le temps d'immersions dans les technologies de l'entreprise.

Mais, il a néanmoins été porté à mon attention que le projet \og Vente en Plus \fg{} devait être effectué comme un véritable projet avec des objectifs précis, notamment en terme de temps. Je pense sincèrement que grâce à l'expérience apportée par Tampon-FLASH, je suis en mesure de répondre à ce nouveau défi.
