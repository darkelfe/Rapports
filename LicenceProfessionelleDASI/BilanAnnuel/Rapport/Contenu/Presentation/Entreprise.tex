\chapter{Entreprise}
\section{\fidit}
Tout commence en 2003. La société Diffusion Télématique et Réseaux (DTR), serveur Minitel et leader Lyonnais de l'internet en 1996, décide de fonder une nouvelle entreprise : \fidit{}.

~

\fidit{} est une société d'hébergement web et de réalisation de sites web en tout genre qui dispose de quinze années d'expérience dans le domaine des réseaux et des télécommunications.

En plus des sites web, \fidit{} possède en parallèle plusieurs systèmes en rapport avec internet et les réseaux : mise en place de serveurs, un logiciel de gestion de projet, etc.

~

En 2006, \fidit{} s'agrandit et se diversifie en prenant le contrôle d'une SSII\footnote{SSII : Société de Services en Ingénierie Informatique.} développant un progiciel sur le transport international et la logistique : \solulog.

\section{\solulog}
\solulog{} est une SSII fondée en janvier 2004 par Philippe \bsc{Perisse}. Elle est basée à Villefontaine, dans la région de l'Isle-d'Abeau (département de l'Isère).

~

Au cours de cette même année, M. \bsc{Perisse} sera rejoint par un autre développeur : Alexandre \bsc{VICQ}.

~

Ensemble, ils vont faire évoluer et maintenir le logiciel \integrale{}\footnote{Il s'agit du logiciel développé et maintenu par \solulog. Veuillez vous référer au chapitre \ref{activitepro_solulog} pour plus de détails.} jusqu'en 2006. C'est cette année-là que va s'établir le partenariat entre \solulog{} et une autre société informatique : \fidit. Celle-ci va alors devenir un actionnaire majoritaire de \solulog{} et prendre officiellement son contrôle. De manière officieuse, la direction de \solulog{} est faite conjointement par les deux dirigeants.

~

Cette association va permettre à \solulog{} de proposer un hébergement sur serveur de son logiciel, et lui donner la possibilité de créer un site web en rapport avec \integrale. La collaboration va également apporter une force commerciale non négligeable.

~

Pour faire face à une charge de travail croissante, \solulog{} recrute en 2008 un troisième développeur en alternance : Samuel \bsc{Raposo}, issu de la licence professionnelle Développement et Administration de Systèmes d'Information (DASI) proposée par l'IUT A de \bsc{Lyon}.

~

L'année 2009 verra l'engagement définitif de Samuel au sein de l'entreprise, tandis qu'Alexandre donnera sa démission.

~

En 2010, \solulog{} a de grandes difficultés à maintenir une vitesse de développement correcte, notamment à cause du départ d'Alexandre. Elle va alors se tourner de nouveau vers l'alternance pour recruter un autre développeur (moi-même), ainsi que deux commerciales.